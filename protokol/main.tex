\documentclass[10pt,a4paper]{report}
\usepackage[czech]{babel}
\usepackage[utf8]{inputenc}
\usepackage[T1]{fontenc}
%\usepackage{showframe}
\usepackage[total={18.5cm, 25cm}, top=3cm, left=1.25cm, includefoot]{geometry}
\usepackage{fancyhdr}
\usepackage{amsmath}
\usepackage{graphicx}
\usepackage{float}
\usepackage{ctable}
\usepackage{multirow}
\usepackage{array}
\usepackage{caption}
\usepackage{subcaption}
\usepackage{etoolbox}

\preto\tabular{\shorthandoff{-}}

\renewcommand{\familydefault}{\sfdefault}
\renewcommand\thesection{}
\renewcommand{\headrulewidth}{0.0pt}


\newcommand{\subject}{Měření v elektrotechnice}
\newcommand{\taskname}{Měření statické hysterezní smyčky}
\newcommand{\tasknum}{7A}

\newcommand{\class}{BK2EST}
\newcommand{\group}{01}
\newcommand{\name}{Ondřej}
\newcommand{\surname}{Povolný}
\newcommand{\support}{}
\newcommand{\datmeas}{21.10.2017}

\pagestyle{fancy}
\fancyhf{}

\fancyfoot[L]{\name~\surname}
\fancyfoot[C]{\thepage/\pageref{end}}
\fancyfoot[R]{\class/\group}


\begin{document}

\setcounter{page}{1}
	\begin{center}
		\begin{tabular}{| p{2.5cm} p{2.5cm}   p{2.5cm}  p{2.5cm}  p{2.5cm}  p{2.5cm} |}
		\hline
			\multicolumn{2}{| c |}{\multirow{8}{*}{\includegraphics[width=5cm]{logo.png}}}	& \small Předmět&	&	&	\\
			&	&\multicolumn{1}{|l}{}	&	& \multicolumn{2}{l|}{\bf\subject}	\\ \cline{3-6}
			&	& \multicolumn{1}{|l}{\small Jméno a příjení}	&	& 	&	\\
			&	&\multicolumn{1}{|l}{}	&	& \multicolumn{2}{l|}{\name ~ \surname}	\\ \cline{3-6}
			&	&  \multicolumn{1}{|l}{\small Ročník}	&	&\multicolumn{1}{|l}{\small Studijní skupina} &	\\
			&	&\multicolumn{1}{|l}{}	&\multicolumn{1}{c|}{\class} & &\multicolumn{1}{c|}{\group}\\ \cline{3-6}
			&	&  \multicolumn{1}{|l}{\small Spolupracoval} &	&\multicolumn{1}{|l}{\small Měřeno dne} &	\\
			&	&\multicolumn{1}{|l}{}	&\multicolumn{1}{c|}{\support} & &\multicolumn{1}{c|}{\datmeas}\\ \hline
			\small Kontroloval	&	&  \multicolumn{1}{|l}{\small Hodnocení} &	&\multicolumn{1}{|l}{\small Dne} &	\\
			&	&\multicolumn{1}{|l}{}	&	&\multicolumn{1}{|l}{}	 	&	\\ \hline
		\multicolumn{1}{|l|}{\small Číslo úlohy} & \small Název úlohy 	& &	&&	\\
		\multicolumn{1}{|c|}{\Huge\bf \tasknum} &\multicolumn{5}{l|}{\Huge \bf \taskname} \\ \hline
		\end{tabular}
	\end{center}


\section{Zadání}
\begin{itemize}
	\item Obeznamte se s měřením křivek prvotní magnetizace a hysterezních smyček integrační metodou při stejnosměrném magnetování
	\item Změřte statickou hysterezní smyčku předložených feromagnetických jader manuálně pomocí digitálního osciloskopu a ověřte jejich chování při změně směru magnetování
	\item Změřte statickou hysterezní smyčku předložených feromagnetických jader pomocí automatizovaného pracoviště
\end{itemize}

\section{Teoreticky úvod}
Hysterezni charakteristika je grafická závislost mezi magnetickou indukci $B$ feromagnetickeho materialu a intenzitou magnetickeho pole $H$. O stejnosmernem magnetovani mluvime tehdy, pokud 
\label{end}
\end{document}
